\documentclass{article}
\usepackage[utf8]{inputenc}
\usepackage{amsmath} 
\usepackage{enumitem}
\usepackage{braket}

\title{Chapter 1 Problems}
\author{William Arnold}
\date\today

\begin{document}
\maketitle 

\section*{Problem 1.1}
For the distribution of ages in the example in Section 1.3.1
\begin{align*}
  N(14) &= 1 \\
  N(15) &= 1 \\
  N(16) &= 3 \\
  N(22) &= 2 \\
  N(24) &= 2 \\
  N(25) &= 5
\end{align*}

\begin{enumerate}[label=(\alph*)]
  \item Compute $\braket{j^2}$ and $\braket{j}^2$
    Total number of kids is $1 + 1 + 3 + 2 + 2 + 5 = 14$. 
    \begin{align*}
      \braket{j^2} &= \frac{1}{14}(14^2 + 15^2) + \frac{2}{14}(22^2 + 24^2) + \frac{3}{14}16^2 + + \frac{5}{14}25^2 \\
                   &= \frac{3217}{7} \approx 459.57  \\
      \braket{j} &= \frac{1}{14}(14 + 15) + \frac{2}{14}(22 + 24) + \frac{3}{14}16 + + \frac{5}{14}25 \\
                   &= 21 \\
      \braket{j}^2 &= 441
    \end{align*}
  \item Determine $\Delta j$ for each $j$ and use Equation 1.11 to compute the standard deviation
    \begin{align*}
      \sigma^2 &= \braket{(\Delta j)^2} = \frac{1}{14}((-7)^2 + (-6)^2) + \frac{2}{14}(1^2 + 3^2) + \frac{3}{14}(-5)^2 + + \frac{5}{14}4^2 \\
               &= \frac{130}{7}
    \end{align*}

  \item Check using part a and part b 
    \begin{align*}
      \sigma^2 &= \braket{j^2} - \braket{j}^2 = \frac{130}{7} \\
               &= \braket{(\Delta j)^2} 
    \end{align*}
\end{enumerate}

\newcommand{\intinf}{\int_{-\infty}^\infty}
\newcommand{\intzinf}{\int_{0}^\infty}

\section{Problem 1.3}
  Consider the Gaussian Distribution
  \[
    \rho(x) = Ae^{-\lambda (x - a)^2}
  \]
  where $A, a$ and $\lambda$ are positive real constants (necessary integrals are in the back of the book)
  \begin{enumerate}[label=(\alph*)]
    \item Use equation 1.16 to determine $A$
      We have that $\intinf \rho(x) = 1$ so we have 
        \begin{align*}
          \intinf Ae^{-\lambda (x - a)^2} dx &= 2A\intzinf e^{-\lambda x^2} dx \\
                                                             &= 2A\sqrt{\pi}\frac1{2\sqrt{\lambda}} \\
                                                             &= A \sqrt{\frac\pi{\lambda}} = 1
        \end{align*}
        Thus solving for $A$ we get
        \[ A = \sqrt{\frac\lambda{\pi}} \]

    \item Find $\braket{x}, \braket{x^2}$, and $\sigma$
     
      Let $u = x-a$. Then $\frac{du}{dx} = 1, du = dx$, and
      \begin{align*}
        \braket{x} &= \intinf Axe^{-\lambda (x - a)^2}dx \\
                   &= \intinf A(u + a)e^{-\lambda u^2}du \\
                   &= \intinf Aue^{-\lambda u^2}du + \intinf Aae^{-\lambda u^2}du \\
      \end{align*}
      Since the first integral in the sum is integrating an odd function ($ue^{\lambda u^2} = -( (-u)e^{\lambda (-u)^2})$), that integral is zero. This leaves
      \begin{align*}
        \braket{x} &= \intinf Aae^{-\lambda u^2}du \\
                   &= a \intinf Ae^{-\lambda u^2}du \\
                   &= a
      \end{align*}
      
      Using the formula in the back of the book, we have that
      \begin{align*}
        \braket{x^2} &= \intinf x^2 Ae^{-\lambda (x - a)^2} dx \\
                     &= \intinf (u + a)^2 Ae^{-\lambda u^2} du \\
                     &= \intinf u^2 A e^{-\lambda u^2}du + \intinf 2auAe^{-\lambda u^2}du + \intinf a^2Ae^{-\lambda u^2}du \\
                     &= 4A\sqrt{\pi}(\frac{1}{2\sqrt{\lambda}})^3  + a^2 \\
                     &= \frac{1}{2} \frac{\sqrt\lambda}{\sqrt\pi} \sqrt{\pi} \frac{1}{\sqrt{\lambda}^3} + a^2 \\
                     &= \frac{1}{2 \lambda} + a^2
      \end{align*}
      And lastly, 
      \begin{align*}
        \sigma^2 = \braket{x^2} - \braket{x}^2 = \frac{1}{2\lambda} + a^2 - a^2 = \frac{1}{2\lambda}
      \end{align*}
  \end{enumerate}

\end{document}
